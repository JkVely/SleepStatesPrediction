\documentclass[conference]{IEEEtran}
\IEEEoverridecommandlockouts
\usepackage{cite}
\usepackage{amsmath,amssymb,amsfonts}
\usepackage{algorithmic}
\usepackage{graphicx}
\usepackage{textcomp}
\usepackage{xcolor}
\usepackage{threeparttable}
\usepackage{float}
\usepackage{hyperref}

\floatstyle{boxed} 
\restylefloat{figure}

\def\BibTeX{{\rm B\kern-.05em{\sc i\kern-.025em b}\kern-.08em
    T\kern-.1667em\lower.7ex\hbox{E}\kern-.125emX}}
\begin{document}

\title{Sleep State Prediction Using Accelerometry Data: A Systematic and Computational Approach}

\author{
	\IEEEauthorblockN{Juan Carlos Quintero Rubiano}
	\IEEEauthorblockA{Code: 20232020172\\
		\textit{Systems Engineering} \\
		\textit{Francisco Jose de Caldas District University}\\
		Bogota, Colombia \\
		jcquineror@udistrital.edu.co}\\
	%and
	\IEEEauthorblockN{Juan Felipe Wilches Gomez}
	\IEEEauthorblockA{Code: 20231020137\\
		\textit{Systems Engineering} \\
		\textit{Francisco Jose de Caldas District University}\\
		Bogota, Colombia \\
		jfwilchesg@udistrital.edu.co}
	\and
	\IEEEauthorblockN{Juan Nicolas Diaz Salamanca}
	\IEEEauthorblockA{Code: 20232020059\\
		\textit{Systems Engineering} \\
		\textit{Francisco Jose de Caldas District University}\\
		Bogota, Colombia \\
		jndiazs@udistrital.edu.co}
}

\maketitle

\begin{abstract}
	Sleep is a complex physiological process composed of several distinct states, each playing a critical role in health and well-being. As a highly sensitive biological system, sleep demonstrates remarkable susceptibility to subtle environmental perturbations, where minor changes in ambient conditions, noise levels, or external stimuli can significantly influence sleep onset, duration, and quality transitions. Traditional measurement of sleep states relies on polysomnography, an expensive and intrusive method. Recent advances in wearable technology have enabled the use of accelerometers as a non-invasive, low-cost alternative for sleep state prediction. This paper presents a systematic analysis and computational approach for predicting sleep states using accelerometry data, focusing on the extraction of movement patterns and temporal features while considering the inherent environmental sensitivity of sleep systems. We discuss the theoretical background, computational models, and methodological considerations, highlighting the potential and limitations of accelerometer-based sleep monitoring in capturing these environmentally-influenced state transitions.
\end{abstract}

\begin{IEEEkeywords}
	Sleep states, accelerometry, computational models, wearable devices, sleep monitoring, time series analysis.
\end{IEEEkeywords}

\end{document}